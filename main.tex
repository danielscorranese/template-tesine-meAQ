\documentclass[12pt]{article} 

% Importa il pacchetto di stile personalizzato
\usepackage{meAQ_Casella}
\usepackage{ragged2e}

% Dati della tesina (inserisci qui i tuoi dati)
\newcommand{\TitoloTesina}{MODELLO DI REDAZIONE DELLE TESINE D'ESAME}
\newcommand{\AutoreTesina}{(nome e cognome)}
\newcommand{\CorsoEsame}{esame del corso di ...}
\newcommand{\DataEsame}{data ...}
\newcommand{\CorsoAccademico}{DCSL34 oppure DCPL34}

% Metadati del PDF (titolo, autore, ecc.)
\usepackage[hidelinks]{hyperref}
\hypersetup{
    pdftitle    = {\TitoloTesina},
    pdfauthor   = {\AutoreTesina},
    pdfsubject  = {\CorsoEsame},
    pdfkeywords = {musica elettronica, tesina, Conservatorio A. Casella}
}

% Importa il file bibliografico
\addbibresource{bibliografia.bib}

\begin{document}

% --- PRIMA PAGINA (Frontespizio)
\begin{titlepage}
    \thispagestyle{plain} % mostra il numero di pagina anche sul frontespizio
    \begin{center}
        % Intestazione
        Corsi Accademici di Musica Elettronica / \CorsoAccademico\\
        Conservatorio A. Casella, L'Aquila
        
        \vspace{0.2cm}
        
        % Titolo tesina
        {\fontsize{14}{17}\selectfont \textbf{\uppercase{\TitoloTesina}}}\\
        % Nome e cognome studente
        di \AutoreTesina
        
        \vspace{0.7cm}
        
        % Dati esame
        \CorsoEsame\\
        \DataEsame
        
        \vspace{9cm}
        
        % Abstract
        \makebox[\linewidth][l]{\textbf{RIASSUNTO} (abstract)}
        
        \vspace{0.5cm}
        \begin{justify}
            Vengono qui esposte alcune linee-guida per la redazione di tesine d’esame per i Corsi 
            Accademici di Musica Elettronica DCPL34 e DCSL34, sia di indirizzo Composizione di
            musica elettroacustica, sia di indirizzo Esecuzione di musica elettroacustica e regia del
            suono. La redazione finale della tesina va formattata secondo le indicazioni date nel
            presente documento e va consegnata sia in versione a stampa sia in versione elettronica 
            (in formato PDF).
            
            La prima pagina (questa pagina) contiene l'intestazione (titolo della tesina, intestazione
            dell'esame e annualità del corso relativo, nome del candidato, data d'esame) e un breve
            riassunto o “abstract” (massimo mezza pagina). Il riassunto si limita a richiamare il tema
            trattato e i punti salienti sviluppati nel testo principale della tesina. Il testo principale inizia
            alla pagina successiva.
            
            In coda al riassunto, è possibile segnalare se il testo è integrato da materiali multimediali
            (audio, software, ecc.). Questi ultimi vanno eventualmente presentati su un CD-rom
            allegato alla copia cartacea, oppure mediante supporto Internet (in questo caso, fornire
            con chiarezza l'URL dove reperire questo materiale). Eventuali partiture possono essere
            fornite in cartaceo, oppure possono essere parte del documento elettronico,
            preferibilmente come Appendice alla fine del testo principale.
        \end{justify}
        
    \end{center}
\end{titlepage}

% --- INIZIO TESTO ---
\setcounter{page}{2} % la pagina successiva al frontespizio è la 2

% --- [SECTION] INTRODUZIONE ---
\section{INTRODUZIONE} 
L’introduzione serve a circoscrivere e identificare il tema affrontato, nonché a descrivere
come lo si è voluto affrontare e come è stato organizzato il discorso complessivo della
tesina. Se il lavoro non ha carattere soltanto compilativo, ma si basa su una propria ricerca
o riflessione originale, o sulla produzione di elaborati tecnici o musicali, l’introduzione
serve a chiarire motivazioni e prospettive metodologiche adottate al proprio fine.

L’introduzione serve in ogni caso a delineare l’ambito di pratiche e di interessi di cultura
musicale e tecnico-scientifica pertinenti nello studio della tematica, nonché delle eventuali
problematiche che la connotano e delle implicazioni generali che essa può rivestire, anche
con riferimento a precedenti contributi di ricerca o di produzione che valgano da corpus di
“conoscenze condivise” nella particolare area di interessi.

\subsection{Struttura e formattazione del testo}
Si usi un'impostazione di pagina tipo A4 (210 x 297 cm), con margini del testo 2,5 cm
(superiore e inferiore) e 2 cm (destro e sinistro). Le pagine vanno numerate.

Come si vede nella pagina precedente, l’intestazione è posta al centro della pagina, con
titolo in carattere Arial corpo 14 grassetto maiuscolo. Per il resto del documento, il testo è
in carattere Arial corpo 12.

Il testo può essere suddiviso in "sezioni" o "parti". Il titolo di sezione è in carattere
grassetto e maiuscolo, come si vede all'inizio di questa sezione intitolata \textbf{INTRODUZIONE}.
Ogni sezione può avere una o più "sottosezioni". Il titolo di sottosezione è in grassetto non
maiuscolo, come si vede in questa sottosezione intitolata \textbf{Formattazione del testo}.

Il testo ha interlinea singola e margini giustificati sia a sinistra che a destra. Gli accapo
lasciano una riga vuota e senza rientro (tutto a sinistra, senza indentazione). Eventuali
note a fondo pagina avranno carattere Arial corpo 10, come in questa nota di esempio.\footnote{Le 
note a fondo pagina vanno inserite correttamente usando le corrispondenti funzioni del programma di
word processing utilizzato.}

% --- [SECTION] ALCUNE CONVENZIONI DI REDAZIONE ---
\section{ALCUNE CONVENZIONI DI REDAZIONE}

% --- [SUBSECTION] TITOLI E ALTRI SOSTANTIVI PARTICOLARI ---
\subsection{Titoli e altri sostantivi particolari}
I titoli di opere musicali o di altre opere artistiche devono essere in corsivo. Per esempio,
\textit{Kontakte} di Karlheinz Stockhausen. Sostantivi come Sonata e locuzioni come Sonata
op.110 oppure Sinfonia n.5 non sono titoli, basta segnarle con iniziale maiuscola. I nomi di
autori, compositori, ecc. vanno dati per esteso alla loro prima ricorrenza (per esempio
Bruno Maderna), mentre basta il cognome in ricorrenze successive (Maderna). I nomi di
persona stranieri devono essere quelli in lingua-madre (Kepler non Keplero, Descartes
non Cartesio). Alla prima ricorrenza del nome di autori che rivestono interesse primario
nella tesina, vanno inseriti anche anno di nascita e di morte, per esempio: Bruno Maderna
(1920-1973). Alla prima occorrenza di un titolo di opera musicale (o di altro tipo) che
riveste interesse primario nella tesina, indicare anche l’anno di composizione o di produzione, 
tra parantesi dopo il titolo, per esempio: \textit{Musica su due dimensioni} (1958). Il richiamo 
a incisioni discografiche viene illustrato più avanti.

Se nel corpo del testo vengono richiamati titoli di libri, essi vanno dati in corsivo come per
le opere d'arte, per esempio: \textit{L’opera d’arte all’epoca della sua riproducibilità tecnica}
(Benjamin 1936). I titoli di articoli vanno invece indicati in carattere normale tra virgolette,
per esempio: "Wiener’s insight into communication" (von Glasersfeld 1994).

Le parole straniere d’uso non comune vanno in carattere corsivo, cioè in italics. Parole
d’uso frequente, soprattutto in contesti specialistici di rilievo primario nella tesina, vanno in
corsivo quando vengono spiegati o tradotti (per esempio: \textit{feedback} in italiano significa
retro-alimentazione), altrimenti possono avere carattere normale (per esempio: il feedback
è condizione normale di un sistema di trattamento di energia o di informazione).

% --- [SUBSECTION] CITAZIONI TESTUALI ---
\subsection{Citazioni testuali}
Le citazioni testuali da articoli, libri o altra fonte bibliografica, vanno inserite nel corpo
principale della tesina, tra virgolette se brevi (meno di tre righe). Se sono più estese,
possono costituire un blocco di testo a parte con corpo 11, iniziando con accapo, indentate
a sinistra. Ogni citazione testuale va seguita da annesso richiamo bibliografico, meglio se
anche con indicazione di pagina. Esempio di citazione breve: "Distinguiamo, innanzitutto, i
due tipi di segnali con cui avremo a che fare: segnali analogici e segnali digitali" (Del Duca
1987, p.17). Esempio di citazione estesa:

\begin{quote}
    \small
    Molti circuiti elettronici analogici, in particolare gli amplificatori, vengono realizzati usando
    schemi a controreazione allo scopo di ottenere buone prestazioni, migliori che in assenza di
    reazione. Sebbene il principio della reazione negativa (negative feedback) fosse noto già da
    tempo (un esempio classico è il regolatore di Watt), la sua introduzione esplicita e la sua
    formalizzazione viene attribuita all'ingegnere americano Harold S. Black, che negli anni '20 lo
    utilizzò per risolvere i problemi di stabilità del guadagno e di distorsione negli amplificatori a
    tubi elettronici per telefonia a grandi distanze. Oltre che negli amplificatori, di cui ci
    occuperemo in quanto segue, la controreazione trova largo impiego nella strumentazione e
    nei sistemi di controllo (Pallottino 2003, p.163).
\end{quote}

Le citazioni in lingua straniera possono essere riportate nella lingua originale, non tradotte.
Oppure possono essere tradotte nel corpo principale del testo, riportando l’originale in una
nota a fondo pagina.

% --- FIGURE ---
\subsection{Figure}
Introdurre solo figure che siano opportune o necessarie a comprendere il flusso del
discorso. Evitare ritratti di persone o elaborazioni grafiche puramente decorative.

\begin{figure}[h]
    \centering
    \begin{minipage}{0.58\textwidth}
        Le figure vanno importate nel documento di testo. Possono essere nel testo (come qui accanto) 
        oppure, se numerose, possono essere raccolte a fondo testo, dopo la bibliografia. 
        Dovrebbero essere di risoluzione grafica non inferiore a 300 dpi e dimensionate in modo congruo. 
        Vanno numerate, col numero posto in didascalia vicino alla figura, e vanno richiamate nel testo 
        facendo riferimento appunto a questa numerazione. Per esempio, in figura~\ref{fig:orecchio} 
        vediamo uno schema anatomico dell'orecchio umano.
    \end{minipage}\hfill
    \begin{minipage}{0.35\textwidth}
        \centering
        \includegraphics[width=\linewidth]{figures/schemaAnatomicoOrecchio.png}
        \caption{}
        \label{fig:orecchio}
    \end{minipage}
\end{figure}

% --- [SUBSECTION] FORMULE ---
\subsection{Formule}
Espressioni matematiche ed equazioni semplici vanno inserite dopo un accapo ed una tabulazione. Per esempio:
\[ Fc = 1/Tc \]

Eventuali apici o pedici, all'interno di un'espressione, devono essere in corpo ridotto:
\[ 2^2 = 4 \]

Formule più complesse possono essere trattate come fossero figure: vanno composte con altro software 
e importate nel documento al punto giusto. Oppure, se si usa LaTeX, possono essere inserite come 
equazioni. Per esempio:
\begin{equation}
    Y[f] = \sum_{n=-\infty}^{+\infty} y[nT]\,e^{-j 2\pi \frac{f} {F_{s}} n}
    \label{eq:dft_fourier}
\end{equation}

Se un'equazione viene richiamata nel testo, occorre numerarla. Per esempio, qui sopra 
la eq \ref{eq:dft_fourier} mostra la trasformata di Fourier a tempo discreto (DFT), mentre 
la eq \ref{eq:integrale_fourier}, qui sotto, illustra l'integrale che ci permette di invertirla.

\begin{equation}
    y(nT) = \frac{1}{F_{s}}\int_{-F_{s}/2}^{F_{s}/2}Y(f)e^{j2\pi fnT}df
    \label{eq:integrale_fourier}
\end{equation}

% --- RICHIAMI BIBLIOGRAFICI NEL TESTO ---
\section{RICHIAMI BIBLIOGRAFICI NEL TESTO}
I richiami bibliografici servono a indicare la fonte di una certa informazione o di un certo
dato rilevante nel proprio discorso, oppure a indicare che un certo argomento è stato
avanzato o sviluppato e approfondito da un certo autore, la cui pubblicazione viene
pertanto presa a riferimento. Un richiamo bibliografico quindi equivale a dire,
implicitamente, "come afferma Tizio” oppure “come si vede nella pubblicazione di Caio del
2012", o ancora "sulla questione si veda ciò che ne scrisse Sempronio nella sua
pubblicazione del 1999".

I richiami bibliografici possono essere inseriti nel corpo principale del testo (o
alternativamente in una nota a fondo pagina) mettendo fra parentesi autore e anno di
pubblicazione, per esempio (\cite{delduca1987}). Se si richiama un passaggio specifico di una
pubblicazione, indicare la pagina (\cite{delduca1987}, p.17) o le pagine (\cite{delduca1987},
pp.17-19). Se vengono effettuati due o più richiami in una stessa circostanza, essi saranno
inseriti nella medesima parentesi in ordine temporale, divisi da punto-e-virgola 
(\cite{delduca1987,vonglasersfeld1994}). Se ci si richiama a una raccolta antologica, comprendente
scritti di autori diversi, si indica il nome del curatore o dei curatori dell’antologia, come
fossero autori (in bibliografia si indicherà “a cura di”). Se il testo richiamato o citato ha due
autori o due curatori, si riporta il nome di entrambi (\cite{bianchini2000}). Se vi sono
tre o più autori o curatori, basta il primo nome seguito dall’abbreviazione \textit{et al.} 
("ed altri", in corsivo perché latino), per esempio (\cite{dannenberg2003}).

Ogni richiamo bibliografico deve corrispondere a uno dei titoli raccolti nella bibliografia
presentata a fine testo.

% --- [SUBSECTION] BIBLIOGRAFIA ---
\subsection{Bibliografia}
L'elenco delle fonti bibliografiche, posizionato a inizio pagina dopo la fine del testo
principale, è dato in ordine alfabetico secondo il cognome degli autori, in carattere Arial
corpo 10. Fonti attribuite a uno stesso autore sono ordinate secondo l’anno di
pubblicazione, dal meno al più recente. Per le antologie e gli atti di convegni, si riporta il
nome del curatore come fosse nome di autore, ma seguito dalla clausola “a cura di”. Si
veda esempio di bibliografia più avanti in questo documento.

Ogni entrata bibliografica segue questo schema generale:
\begin{quote}
    \small
    Cognome, Nome (anno di pubblicazione). Titolo, editore (eventuali ulteriori dati bibliografici).
\end{quote}

Il titolo di libri (monografie) è in corsivo, seguito luogo di edizione ed editore (vale anche
per lavori di tesi e dissertazioni dottorali). Esempi:
\begin{quote}
    \small
    Del Duca, Massimo (1987). \textit{Musica digitale. Sintesi analisi e filtraggio digitale nella musica elettronica},
    Padova, Franco Muzzio Editore.

    Bianchini, Riccardo e Alessandro Cipriani (2000). \textit{Virtual Sound}, Roma, Contempo Edizioni.

    Appleton, Jon e Ricardo Perera, a cura di (1975). \textit{The Development and Practice of Electronic
    Music}, Englewood Cliffs, Prentice-Hall.
\end{quote}

Il titolo di articoli o saggi contenuti in volumi antologici va tra virgolette, seguito dal titolo
del libro (tra parentesi nome dei curatori, con la dicitura "a cura di"), dal luogo di edizioni e
dall’editore. Può essere utile indicare le pagine corrispondenti all'articolo.
\begin{quote}
    \small
    Bernardini, Nicola (1986). “Live electronics”, in \textit{Nuova Atlantide. Il continente della musica elettronica}
    1900–1986 (a cura di Roberto Doati e Alvise Vidolin), La Biennale di Venezia, Vallecchi, pp. 61–78.

    Emmerson, Simon (2012). “Live Electronic Music or Living Electronic Music?” in \textit{Bodily Expression in
    Electronic Music: Perspectives on Reclaiming Performativity} (a cura di D. Peters, G. Eckel e A.
    Dorschel), Londra, Routledge, pp. 152–162.
\end{quote}

Il titolo di articoli apparsi in rivista o in altre pubblicazioni periodiche è tra virgolette, seguito
da titolo della rivista in corsivo e numero dell’annata della pubblicazione. Può essere utile
aggiungere, infine, le pagine corrispondenti all'articolo.
\begin{quote}
    \small
    Von Glasersfeld (1994). "Wiener’s Insight into Communication", \textit{Kybernetes}, vol.23, n.7, pp.21-22.
\end{quote}

Se viene presa a riferimento la traduzione italiana di un testo precedentemente pubblicato
in altra lingua, è buona norma aggiungere l'anno di pubblicazione dell'originale. Per
esempio:
\begin{quote}
    \small
    Benjamin, Walter (1966). \textit{L’arte all’epoca della sua riproducibilità tecnica}, Torino, Einaudi (edizione
    originale tedesca 1936).
\end{quote}

Lo stesso vale per opere antiche e molto antiche, indicando la data di pubblicazione
moderna e segnalando la data di redazione originale. Per esempio:
\begin{quote}
    \small
    Agostino di Ippona (1969), \textit{De Musica}, Firenze, Sansoni (redazione originale ca. 387 dC)
\end{quote}

% --- [SUBSECTION] RICHIAMI SITOGRAFICI E SITOGRAFIA ---
\subsection{Richiami sitografici e sitografia}
Laddove ci si richiama ad un sito web, si indica l’autore o (in assenza di questo) il nome
identificativo del sito e la data di redazione della pagina web corrispondente, per esempio
(Tanzi 2005). Si sconsiglia di fare riferimento a pagine web che non riportano l’anno di
pubblicazione; tuttavia, se proprio necessario, in quel caso si potrà indicare la data del
proprio accesso al sito, comprovante l'esistenza del sito stesso almeno a quella data.

La sitografia viene presentata a fine testo, sempre in carattere Arial corpo 10, come per la
bibliografia. Essa si presenta come lista ordinata alfabeticamente secondo i cognomi degli
autori richiamati, seguiti dall’anno di pubblicazione (o dalla data del proprio ultimo accesso
al particolare sito), dal titolo (se presente) e soprattutto dall’URL (universal resource
locator, indirizzo internet).

Si veda esempio di sitografia in questo stesso documento.

% --- [SUBSECTION] RICHIAMI DISCOGRAFICI E DISCOGRAFIA ---
\subsection{Richiami discografici e discografia}
Nel testo è possibile fare richiami a incisioni discografiche o ad altre fonti sonore, da
trattare come richiami bibliografici o sitografici. Si può usare la dicitura "vedere discografia"
e introdurre debitamente a fine testo un elenco discografico, in ordine alfabetico con chiari
riferimenti editoriali (luogo di pubblicazione, etichetta, numero di serie).

Si vedano esempi di discografia alla fine di questo.

% --- [SECTION] CONCLUSIONE ---
\section{CONCLUSIONE}
E’ buona norma concludere il testo riassumendo l’ambito degli argomenti e delle
osservazioni presentate, il significato di quanto discusso (per esempio il rilievo storico di
un lavoro musicale nel contesto di una certa tradizione o di un certo repertorio o di un
certo approccio tecnologico, oppure il potenziale estetico di una certa tecnica o prassi
artistica). Inoltre si possono indicare gli elementi tralasciati ma che meriterebbero
approfondimenti ulteriori, oppure le difficoltà riscontrate nel trattare certi aspetti. Indicare
anche se ci sono e quali sono gli eventuali elementi di raccordo fra il lavoro presentato e
altri elaborati d'esame sviluppati nel proprio percorso di formazione.

% --- BIBLIOGRAFIA ---
\newpage
% Stampa la bibliografia
\printbibliography[title={BIBLIOGRAFIA}]
\nocite{*} % [FACOLTATIVO] per stampare tutta la bibliografia anche se non citata

% --- SITOGRAFIA E DISCOGRAFIA ---
% Nota: Biblatex può gestire anche queste, ma per semplicità e controllo totale
% spesso conviene farle manualmente se sono poche, o usare filtri avanzati.
% Qui un esempio manuale per rispettare esattamente l'ordine del PDF.

% --- SITOGRAFIA ---
\section{SITOGRAFIA}
{\footnotesize
Tanzi, D. (2005). "Musical objects and digital domains", relazione presentata alla conferenza Electroacoustic
Music Studies Network, Montréal 2005, http://www.ems-network.org/IMG/EMS2005-Tanzi.pdf

Novati, Maddalena (ultimo accesso 10 Febbraio 2021) http://fonologia.lim.di.unimi.it/introduzione.php
}

% --- DISCOGRAFIA ---
\section{DISCOGRAFIA}
{\footnotesize
Nono, L. \textit{Prometeo}. EMI Classic CRMCD 1039, 1993.

Stockhausen, K.H. \textit{Kontakte}. Stockhausen Verlag, 1991.
}

\end{document}